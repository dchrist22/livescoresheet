\documentclass[]{scrartcl}
\usepackage[a4paper, left=2cm, right=5cm, top=0.5cm,landscape]{geometry}
\usepackage{fontspec}
\usepackage{polyglossia}
\usepackage[dvipsnames,table]{xcolor}
\usepackage{booktabs}
\usepackage{ulem}
\definecolor{failred}{RGB}{255,60,60}

\newcommand{\renderliftcell}[2]{
  \ifnum #1>0
    \cellcolor{green} #2
  \else
    \ifnum #1<0
      \cellcolor{failred} \sout{#2}
    \else
      ---
    \fi
  \fi
}

\begin{document}
\section*{LM Kreuzheben NRW}
WEBSTART

CONTENTSTART
\subsection*{KLASSE}
\begin{tabular}{lp{4cm}p{5cm}llrrrrr}
  \midrule
  \textbf{Platz} & \textbf{Name} & \textbf{Verein} & \textbf{Alter} &
  \textbf{Gewicht} & \textbf{1. Versuch} & \textbf{2. Versuch} & \textbf{3.
  Versuch} & \textbf{Total} & \textbf{Punkte} \\
  \midrule
LIFTERS
\end{tabular}
CONTENTEND

LIFTERSTART
PLACE & NAME & CLUB & AGE & BW & \renderliftcell{GOOD1}{ATTEMPT1} &
\renderliftcell{GOOD2}{ATTEMPT2} & \renderliftcell{GOOD3}{ATTEMPT3} & TOTAL & WILKS\\
\midrule
LIFTEREND

WEBEND
\end{document}
